\chapter{Introduzione}
\label{chap:introduzione}
In una società sempre più digitalizzata e dinamica, spesso viene a crearsi un netto distacco tra i settori capaci di \textbf{evolvere assieme alla tecnologia}, e quelli che, per un motivo o per l'altro, rimangono ancorati a procedure e metodologie tradizionali. Il settore medico, in continuo rinnovamento sul lato scientifico, è, soprattutto in Italia, affidato ad \textbf{infrastrutture informatiche progettate tempo fa.} Questo, soprattutto per motivi di stabilità e affidabilità: gli errori, qui, non sono ammissibili. \\
Per questo motivo, spesso non si notano le evidenti possibilità di miglioramenti che la ricerca informatica potrebbe apportare agli ospedali, agli ambulatori, agli studi.
L'obiettivo di questa tesi è mettere luce su una delle possibili modalità con cui l'informatica potrebbe, in un futuro prossimo, migliorare la praticità ma soprattutto la sicurezza degli ambienti ospedalieri.\\
La procedura di triage, ossia il processo di selezione dei pazienti richiedenti cure, è oggi affidata totalmente ad infermieri. Questa scelta è dovuta, oltre ad un evidente bisogno di poter osservare il paziente, alla necessità dell'\textbf{immediatezza di un contatto verbale} con il personale sanitario. \\
Perciò, la sfida nella realizzazione della tesi è soprattutto legata al rendere il \textbf{più umana possibile} l'interazione con un bot automatizzato. Il bot deve quindi accogliere il paziente, comprenderne le problematiche ed i sintomi, e farlo sentire protetto. Non si escluderà del tutto un apporto umano: gli infermieri sono addestrati per poter osservare e comprendere il richiedente cura, e l'apporto dell'osservazione diretta del paziente è ancora troppo importante per escluderla. È senz'altro possibile, però, affidare la prima parte di \textbf{profilazione dell'utente} a procedure automatizzate.
