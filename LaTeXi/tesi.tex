%CLASSE DOCUMENTO - LINGUA E DIMENSIONE FONT
\documentclass[trieste,corpo=11pt,numerazioneromana]{toptesi}

%%%%%%%%%%%%%%%%%%%%%%%%%%%%%%%%%%%%%%%%%%%%%%%%%%%%%%%%%%%%%%%

% INCLUSIONE PACCHETTI
\usepackage[classica]{topfront}
\usepackage[utf8]{inputenc} %utf8
\usepackage[italian]{babel}
\usepackage[T1]{fontenc}
\usepackage{blindtext}
\usepackage{graphicx,wrapfig}
\usepackage{booktabs}
\usepackage{lmodern}
\usepackage{varioref}
\usepackage{url}
\usepackage{array}
\usepackage{paralist}{\obeyspaces\global\let =\space}
\usepackage{verbatim}
\usepackage{subfig}
\usepackage{tabularx}
\usepackage{amsmath}
\usepackage{amsfonts}
\usepackage{float}
\usepackage{float}
\usepackage{amssymb}
\usepackage{multicol}
\usepackage{multirow}
\usepackage{listings}
\usepackage[pass]{geometry}
\usepackage[figuresright]{rotating}
\usepackage{algorithm}
\usepackage{algorithmic}
\usepackage{wrapfig}
\usepackage{amsmath}
\usepackage[babel]{csquotes}
\usepackage{hyperref}
\usepackage[backend=biber,bibencoding=ascii]{biblatex}


%%%%%%%%%%%%%%%%%%%%%%%%%%%%%%%%%%%%%%%%%%%%%%%%%%%%%%%%%%%%%%%

% CONFIGURAZIONE LINK E RIFERIMENTI
\hypersetup{%
    pdfpagemode={UseOutlines},
    bookmarksopen,
    pdfstartview={FitH},
    colorlinks,
    linkcolor={black}, %COLORE DEI RIFERIMENTI AL TESTO
    citecolor={blue}, %COLORE DEI RIFERIMENTI ALLE CITAZIONI
    urlcolor={blue} %COLORI DEGLI URL
}

%%%%%%%%%%%%%%%%%%%%%%%%%%%%%%%%%%%%%%%%%%%%%%%%%%%%%%%%%%%%%%%

% CONFIGURAZIONE LISTATI/CODICE - CANCELLARE SE NON NECESSARIO
% PYTHON - BIANCO E NERO
\lstset{%
	captionpos=b,
	language=Python,
	basicstyle =\small\ttfamily,
	keywordstyle=\color{black}\bfseries,
	breaklines=true,
	breakatwhitespace=true,
	frame=lines,
	numbers=left,
	numberstyle=\footnotesize,
}

%%%%%%%%%%%%%%%%%%%%%%%%%%%%%%%%%%%%%%%%%%%%%%%%%%%%%%%%%%%%%%%

% FRENCHSPACING VA _SEMPRE_ ABILITATO PER DOCUMENTI IN ITALIANO
\frenchspacing

%%%%%%%%%%%%%%%%%%%%%%%%%%%%%%%%%%%%%%%%%%%%%%%%%%%%%%%%%%%%%%%

%DEFINIZIONE SEZIONI IN NUMERAZIONE ROMANA
%ELENCO DEI LISTATI/CODICI
\makeatletter
\newcommand\listofcodes{%
 \iffrontmatter\else\frontmattertrue\fi
 \if@openright\cleardoublepage\else\clearpage\fi
 % change the meaning of \chapter in a group
 \begingroup\def\chapter##1{\@schapter}
 \phantomsection % for the hyperlink
 \addcontentsline{toc}{chapter}{Elenco dei listati}
 \lstlistoflistings
 \endgroup
}
\makeatother

\addto\captionsitalian{%
  \renewcommand{\lstlistlistingname}{Elenco dei listati}%
  \renewcommand{\lstlistingname}{Listato}%
}

%%%%%%%%%%%%%%%%%%%%%%%%%%%%%%%%%%%%%%%%%%%%%%%%%%%%%%%%%%%%%%%

% INFORMAZIONI PDF - PERSONALIZZARE
\pdfinfo{%
  /Title    (Realizzazione di un assistente vocale per il triage ospedaliero)
  /Author   (Simone Montali)
  /Subject  (Triage semi-automatizzato per la sicurezza e la tempestività)
  /Keywords (Triage ML Mycroft Ospedale Hospital)
}

%%%%%%%%%%%%%%%%%%%%%%%%%%%%%%%%%%%%%%%%%%%%%%%%%%%%%%%%%%%%%%%

% LISTA DEI CAPITOLI DA INCLUDERE - PERSONALIZZARE
\includeonly{%
introduzione,%
stateofart,%
app_a,%
}

% FILE DI BIBLIOGRAFIA
\addbibresource{bibliography.bib}

%%%%%%%%%%%%%%%%%%%%%%%%%%%%%%%%%%%%%%%%%%%%%%%%%%%%%%%%%%%%%%%

% INIZIO DOCUMENTO
\begin{document}

%%%%%%%%%%%%%%%%%%%%%%%%%%%%%%%%%%%%%%%%%%%%%%%%%%%%%%%%%%%%%%%

% FRONTESPIZIO - PERSONALIZZARE
% ELIMINATE LE VOCI CHE NON VI SERVONO

% UNIVERSITA - NOME
\ateneo{Università degli studi di Parma}

% FACOLTA - DICITURA - CANCELLARE O DECOMMENTARE
%\FacoltaDi{Faculty of}
% FACOLTA - NOME
\facolta{Ingegneria}

% CORSO DI LAUREA - DICITURA (MANTENERE LO SPAZIO) - CANCELLARE O DECOMMENTARE
%\CorsoDiLaureaIn{Master of Science in }
% CORSO DI LAUREA - NOME
\corsodilaurea{Ingegneria dei Sistemi Informativi}

% TIPOLOGIA TESI
\TesiDiLaurea{Tesi di Laurea di primo livello}

% TITOLO
\titolo{Realizzazione di un assistente vocale per il triage ospedaliero}

% SOTTOTITOLO
\sottotitolo{Triage semi-automatizzato per la sicurezza e la tempestività}

% RELATORE/I - DICITURA - CANCELLARE SE UN SOLO RELATORE
%\AdvisorName{Relatori}
% RELATORE - PROF. NOME E COGNOME
\relatore{prof.ssa\ Mordonini Monica}
% RELATORE AGGIUNTIVO - PROF NOME E COGNOME
% SE SI HA SOLO UN RELATORE ELIMINARE INSIEME AD AdvisorName
\secondorelatore{prof.\ Tomaiuolo Michele}
\terzorelatore{prof.\ Angiani Giulio}
% CANDIDATO - DICITURA (MANTENERE I DUE PUNTI) - CANCELLARE O DECOMMENTARE
%\CandidateName{Candidate:}

% CANDIDATO - NOME E COGNOME
\candidato{Simone Montali}[288144]

% LOGO UNIVERSITA
\logosede{images/logo}

% DATA - MESE ANNO
\sedutadilaurea{Probabilmente 2020}

\frontespizio

%%%%%%%%%%%%%%%%%%%%%%%%%%%%%%%%%%%%%%%%%%%%%%%%%%%%%%%%%%%%%%%

%INTERLINEA - DEFAULT 1 - NON ESAGERATE, NON SUPERATE MAI 1.3 ;)
%\interlinea{1.2}

%%%%%%%%%%%%%%%%%%%%%%%%%%%%%%%%%%%%%%%%%%%%%%%%%%%%%%%%%%%%%%%

\frontmatter

% DEDICA - PERSONALIZZARE
% VSPACE - PROPORZIONE USATA PER CENTRATURA VERTICALE DEL TESTO
% FLUSHRIGHT - ALLINEAMENTO ORIZZONTALE A DESTRA
\vspace*{\stretch{1}}
\begin{flushright}
  \noindent
  Dedica toccante.
\end{flushright}
\vspace*{\stretch{6}}
\cleardoublepage


% CITAZIONE - PERSONALIZZARE
% VSPACE - PROPORZIONE USATA PER CENTRATURA VERTICALE DEL TESTO
% FLUSHRIGHT - ALLINEAMENTO ORIZZONTALE A DESTRA
\vspace*{\stretch{1}}
\begin{flushright}
  \noindent
  Citatemi dicendo che sono stato citato male.

  \textit{Groucho Marx}
\end{flushright}
\vspace*{\stretch{6}}
\cleardoublepage

%%%%%%%%%%%%%%%%%%%%%%%%%%%%%%%%%%%%%%%%%%%%%%%%%%%%%%%%%%%%%%%

% RINGRAZIAMENTI - PERSONALIZZARE
\ringraziamenti
Grazie

%%%%%%%%%%%%%%%%%%%%%%%%%%%%%%%%%%%%%%%%%%%%%%%%%%%%%%%%%%%%%%%

% ABSTRACT - PERSONALIZZARE
\sommario
Abstract della tesi
%%%%%%%%%%%%%%%%%%%%%%%%%%%%%%%%%%%%%%%%%%%%%%%%%%%%%%%%%%%%%%%

% INDICI - ELIMINARE GLI INDICI NON NECESSARI

% INDICE GENERALE
\tableofcontents

% INDICE DELLE FIGURE
\listoffigures

% INDICE DELLE TABELLE
%\listoftables

% INDICE DEI CODICI
%\listofcodes

%%%%%%%%%%%%%%%%%%%%%%%%%%%%%%%%%%%%%%%%%%%%%%%%%%%%%%%%%%%%%%%

\mainmatter

% INCLUSIONE FILE CAPITOLI - PERSONALIZZARE - TENERE COERENTE CON LISTA IN ALTO
\chapter{Introduzione}
\label{chap:introduzione}
In una società sempre più digitalizzata e dinamica, spesso viene a crearsi un netto distacco tra i settori capaci di \textbf{evolvere assieme alla tecnologia}, e quelli che, per un motivo o per l'altro, rimangono ancorati a procedure e metodologie tradizionali. Il settore medico, in continuo rinnovamento sul lato scientifico, è, soprattutto in Italia, affidato ad \textbf{infrastrutture informatiche progettate tempo fa.} Questo, soprattutto per motivi di stabilità e affidabilità: gli errori, qui, non sono ammissibili. \\
Per questo motivo, spesso non si notano le evidenti possibilità di miglioramenti che la ricerca informatica potrebbe apportare agli ospedali, agli ambulatori, agli studi.
L'obiettivo di questa tesi è mettere luce su una delle possibili modalità con cui l'informatica potrebbe, in un futuro prossimo, migliorare la praticità ma soprattutto la sicurezza degli ambienti ospedalieri.\\
La procedura di triage, ossia il processo di selezione dei pazienti richiedenti cure, è oggi affidata totalmente ad infermieri. Questa scelta è dovuta, oltre ad un evidente bisogno di poter osservare il paziente, alla necessità dell'\textbf{immediatezza di un contatto verbale} con il personale sanitario. \\
Perciò, la sfida nella realizzazione della tesi è soprattutto legata al rendere il \textbf{più umana possibile} l'interazione con un bot automatizzato. Il bot deve quindi accogliere il paziente, comprenderne le problematiche ed i sintomi, e farlo sentire protetto. Non si escluderà del tutto un apporto umano: gli infermieri sono addestrati per poter osservare e comprendere il richiedente cura, e l'apporto dell'osservazione diretta del paziente è ancora troppo importante per escluderla. È senz'altro possibile, però, affidare la prima parte di \textbf{profilazione dell'utente} a procedure automatizzate.

\chapter{Stato dell'arte}
\label{chap:stateofart}
Impara l'arte e mettila da parte.


\appendix
% INCLUSIONE APPENDICI - - PERSONALIZZARE - TENERE COERENTE CON LISTA IN ALTO
\chapter{An appendix}
\label{app:a}
% DA RIMUOVERE - LOREM IPSUM PER DIMOSTRAZIONE
\foreignlanguage{english}{\Blindtext}


%%%%%%%%%%%%%%%%%%%%%%%%%%%%%%%%%%%%%%%%%%%%%%%%%%%%%%%%%%%%%%%

% BIBLIOGRAFIA
\phantomsection
\addcontentsline{toc}{chapter}{\refname}
\nocite{*}
\printbibliography

\end{document}
