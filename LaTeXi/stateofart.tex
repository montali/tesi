\chapter{Stato dell'arte}
\label{chap:stateofart}
Illustriamo ora lo stato dell'arte delle varie tecniche e tecnologie che coinvolgono il bot.
\section{Machine Learning}
Il machine learning (in seguito, ML) è una disciplina che si occupa di rispondere, sostanzialmente, a due domande: \textit{Come si può costruire un sistema informatico che migliora con l'esperienza?} e \textit{Quali sono le leggi teoriche fondamentali che governano ogni sistema di apprendimento, sia esso implementato nei computer, negli umani, nelle organizzazioni?}. Il ML copre una serie di compiti diversi, dalla classificazione di email spam, al riconoscimento facciale, al controllo di robot \cite{book:mitchell1997machine}. Ogni problema di ML può essere riassunto come il miglioramento di una determinata performance (data, ad esempio, dall'errore sulle predizioni) durante l'esecuzione di un'operazione. Seguendo l'esempio delle email spam, un algoritmo di ML potrebbe apprendere, tramite un \textit{training set} di email già etichettate come spam/non-spam, a categorizzarne altre.
Portando il problema in termini matematici, vogliamo, dato un grafico
% TODO: crea grafico con punti rossi/blu per spammy/non-spammy
in cui i punti rossi sono email spam, e quelli blu no, trovare una funzione che ci permetta di evidenziare le email da eliminare. Ciò che faremo sarà quindi definire un insieme di \textbf{features}, consistenti, ad esempio, nel contenuto testuale della mail (vettorizzato in termini numerici), l'orario di arrivo, la lunghezza in parole\dots Oltre a ciò, definiremo una \textbf{label}, ossia la caratteristica che vogliamo predirre: sarà, per esempio, 1 se l'email è spam, 0 altrimenti. Ora, il problema consiste in una semplice regressione lineare: date le features $x_0, x_1, x_2$ e la label $y$, vogliamo trovare dei pesi $w_0, w_1, w_2$ tali che
\begin{displaymath}
    w_0x_0+w_1x_1+w_2x_2
\end{displaymath}
approssimi al meglio $y$. Per fare ciò, procediamo per iterazioni: partiremo con dei pesi arbitrari, calcolando $y$ per ogni esempio (già etichettato) fornito, e ne calcoleremo l'errore rispetto alla $y$ reale. Potremo quindi calcolare il gradiente (ossia la somma delle derivate parziali sui rispettivi pesi) della funzione che lega l'errore ai pesi $w_i$, ottenendone un'informazione fondamentale: indicherà infatti la direzione nella quale la funzione errore è decrescente. Ci potremo quindi spostare, di una quantità pari al \textit{learning rate} (uno degli iperparametri definiti arbitrariamente), nella direzione di diminuzione dell'errore. Scegliere un corretto \textit{learning rate} è una parte fondamentale dello sviluppo dell'algoritmo di ML: se è troppo piccolo, la soluzione richiederà troppo tempo, se troppo grande, la soluzione non sarà abbastanza precisa. Dopo un sufficiente numero di esecuzioni dell'algoritmo, il nostro grafico avrà ora questo aspetto:
% TODO: Grafico categorizzato
e saremo quindi in grado di determinare, con una determinata \textit{accuracy}, l'appartenenza di un esempio ad una classe o all'altra.
\section{Assistenti vocali}
Gli assistenti vocali sono software in grado di interpretare dialoghi umani e rispondere attraverso voci sintetizzate. Alcuni esempi famosi sono Apple Siri, Amazon Alexa, Google Assistant. Gli utenti possono effettuare domande, controllare dispositivi domotici, e svolgere un numero di operazioni in continua espansione \cite{article:introduction_to_va}.Il software è costantemente in attesa di una \textit{wake word}; una volta riconosciuta questa parola, si mette in ascolto di comandi. La richiesta viene quindi tradotta in testo tramite un motore di \textit{Speech To Text} ed interpretata dal software. Questa è senz'altro la parte più complessa: il compito di interpretazione di linguaggio naturale è reso complesso dalle svariate sfaccettature che ogni diversa lingua presenta. Interpretata la richiesta, l'assistente procede con le operazioni collegate, e prepara una risposta testuale da dare all'utente. Questa risposta verrà poi sintetizzata in audio da un motore di \textit{Text To Speech}. I moderni dispositivi di assistenza vocale come il Google Home o l'Amazon Echo si appoggiano a server in cloud per l'interpretazione delle richieste: questo permette di ridurne notevolmente le dimensioni ed i requisiti tecnici.
Lo sviluppo, negli ultimi anni, di nuove tecniche di ML, unite ad un grande incremento delle potenze di calcolo e alla disponibilità di vastissimi \textit{dataset linguistici}, hanno permesso grandi miglioramenti nella tecnologia, ma soprattutto l'applicazione ad ambiti, fino a qualche anno fa, impensabili.
\subsection{Assistenti vocali in ambito medico}
Negli anni sono emerse alcune possibili applicazioni delle tecnologie per assistenti vocali in ambito medico. La startup \textbf{Saykara} ha creato un software che permette di ascoltare le conversazioni medico-paziente, allo scopo di creare schede cliniche accurate senza dover impegnare il medico in lunghe procedure di \textit{data entry}.
Un report \cite{article:voicebot_research} di Voicebot, organizzazione per la diffusione degli assistenti vocali, afferma che il 52\% degli intervistati sarebbe intenzionato ad utilizzare assistenti vocali in ambito medico nel futuro. Solo un 7.5\% (1 su 13) l'ha già fatto.
\begin{figure}[H]
    \begin{center}
        \includegraphics[width=0.8\columnwidth]{images/voice-assistant-interest-healthcare.png}
    \end{center}
    \caption{Percentuali di utenti che hanno utilizzato/vorrebbero utilizzare assistenti vocali in ambito medico}
    \label{fig:consumer-interest}
\end{figure}
È interessante notare come anche le fasce di età più avanzate siano, seppur con percentuali più basse, interessate a provare simili applicazioni della tecnologia.
\begin{figure}[H]
    \begin{center}
        \includegraphics[width=0.8\columnwidth]{images/consumer-interest-in-voice-assistants-for-healthcare-use-cases-by-age.png}
    \end{center}
    \caption{Interesse dei consumatori negli assistenti vocali in ambito medico, per fasce di età}
    \label{fig:consumer-interest}
\end{figure}
Un altro esempio interessante di applicazione della tecnologia in ambito medico è \textit{Dragon Medical Virtual Assistant} rilasciato da \textbf{Nuance}, una suite di strumenti vocali capaci di creare cartelle cliniche, eseguire diagnostica sul paziente, fornire risultati in diretta di ricerche mediche, supportare le operazioni di radiologia.